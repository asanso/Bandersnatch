\documentclass{amsart}


\usepackage[capitalize]{cleveref}
\usepackage{tikz}
\usepackage[all]{xy}
\usepackage[foot]{amsaddr}
%\newcommand{\AS}[1]{\color{red}#1\color{black}}

%%%%%%%%%%%%%%%%%%%%%%%%%%%%%%%%%%%%%%%%%%%%%%%%%%
%% Article 
%%%%%%%%%%%%%%%%%%%%%%%%%%%%%%%%%%%%%%%%%%%%%%%%%%

\begin{document}
\title[Bandersnatch]{Bandersnatch: a fast elliptic curve built over the BLS12-381 scalar field.}
\author{Simon Masson}
\address{Heliax}
\author{Antonio Sanso}
\address{Ethereum Foundation and Ruhr Universit{\"a}t Bochum}


\maketitle
\medskip
\begin{abstract}
 In this short note we  introduce Bandersnatch a new elliptic curve built over the BLS12-381 scalar field. The curve is similar to JubJub but is equipped with the GLV endomorphism \cite{10.1007/3-540-44647-8_11} hence it has faster scalar multiplication.
 \end{abstract}

\section{Introduction} 
TODO



\paragraph*{Acknowledgments.} we would like to thank Luca De Feo, Justin Drake, Dankrad Feist, Daira Hopwood and Zhenfei Zhang for fruitful discussions.

\bibliography{bandersnatch}{}
\bibliographystyle{unsrt}
\end{document}