\documentclass{article}

\usepackage{hyperref}
%\usepackage[capitalize]{cleveref}
\usepackage{tikz}
\usepackage[all]{xy}
%\usepackage[foot]{amsaddr}
\usepackage{amsmath,amsfonts,amssymb,amsthm}
\usepackage{xcolor}
\usepackage{tabularx}
\usepackage{booktabs}

%% \newcommand{\AS}[1]{\color{red}#1\color{black}}
\newcommand{\SM}[1]{\color{blue}#1\color{black}}

\newcommand{\Q}{\ensuremath{\mathbb Q}}
\newcommand{\Z}{\ensuremath{\mathbb Z}}
\newcommand{\Fp}{\ensuremath{\mathbb F_p}}
\newcommand{\End}{\ensuremath{\text{End}}}

\newtheorem*{remark}{Remark}

\title{Bandersnatch: a fast elliptic curve built over the BLS12-381 scalar field.}
\author{Simon Masson and Antonio Sanso}
%\address{Heliax}
%\author{Antonio Sanso}
%\address{Ethereum Foundation and Ruhr Universit{\"a}t Bochum}

\begin{document}

\maketitle
\medskip
\begin{abstract}
 In this short note, we introduce Bandersnatch a new elliptic curve
 built over the BLS12-381~\cite{bls12381} scalar field. The curve is
 similar to Jubjub~\cite{jubjub} but is equipped with an efficient
 endomorphism, allowing a fast scalar multiplication algorithm.
 \end{abstract}

\section{Introduction}
%BLS12-381
BLS12-381 is a pairing friendly curve created by Sean Bowe in
2017~\cite{bls12381}.
Currently, BLS12-381 is universally used for digital
signatures and zero-knowledge proofs by many project orbiting in the
blockchin universe: Zcash, Ethereum 2.0, Skale, Algorand, Dfinity,
Chia, and more.
%jubjub
The ZCash team also introduced a new curve built over the BLS12-381
scalar field Jubjub~\cite{jubjub}: a twisted Edwards curve that can
be made efficient inside of the zk-SNARK circuit.
In order for some cryptographic application to scale, it is needed to
have a curve like Jubjub but with a faster scalar multiplication.
One efficient way to speed scalar multiplication up is to employ the
celebrated GLV method~\cite{C:GalLamVan01} (also used by the “Bitcoin
curve” - secp256k1).
This technique was until few months ago protected by a US Patent that
is now expired and freely usable.

\paragraph{Our contribution.}
The Jubjub curve is a curve with a large discriminant, meaning that
the GLV method is not possible on this curve.
We performed an exhaustive search of curves of small discriminant,
defined over the BLS12-381 scalar field. This way, we obtain an
elliptic using the Complex Multiplication (CM) method, and an
efficient scalar multiplication using the GLV method.

\paragraph{Organization of the paper.}
In Section~\ref{sec:small-disc-curves}, we describe how we obtained
several curves allowing the GLV method, together with cryptographic
security.
Then, we introduce in Section~\ref{sec:bandersnatch} the Bandersnatch
curve in different models (in Weierstrass, Montgomery and Twisted Edwards
coordinates).
Finally, we compare the scalar multiplication algorithm over
the Bandersnatch and the Jubjub curves in
Section~\ref{sec:comparison}.


\section{Small discriminant curves}\label{sec:small-disc-curves}

The GLV method~\cite{C:GalLamVan01} is a well known trick for accelerating the
scalar multiplication over particular curve. In a nutshell, it applies
to elliptic curves endowed with an efficiently computable endomorphism
$\psi$.
Generally, the $j=0$ and $1728$ curves are used because non-trivial
automorphisms fit very well, but it also applies for other curves, as
we will see in this section.

% notations
Let $E$ be an elliptic curve defined over $\Fp$. The Frobenius
map $(x,y)\mapsto (x^p,y^p)$ is an endomorphism of degree $p$ of the
curve and we denote $\xi(X) = X^2 - tX + p$ its minimal polynomial.
The coefficient $t$ is called the trace of the curve and is closely
related to the curve order: $\#E(\Fp) = p+1-t$.
In this work, we are looking for ordinary elliptic curves
corresponding to the case where $\End(E)$ is an order of the imaginary
quadratic field $\Q(\sqrt{t^2-4p})$.
We denote $-D$ to be the discriminant of $\End(E)$, and $\{\text{Id},\psi\}$
a basis of the endomorphism ring. The fundamental discriminant
corresponds to the discriminant of the maximal order containing $\End(E)$.
This way, $\psi$ is a degree $\frac{D+1}4$ or $D/4$ depending on the
value of $p$ modulo $4$, and $\psi$ can be defined using polynomials
of degree $O(D)$ thanks to the Vélu's formulas~\cite{velu71}.
Thus, $\psi$ is efficiently computable only for curves of small discriminant.
For example, for $p\equiv 1$ mod $3$, elliptic curves of $j$-invariant $0$
have discriminant $-3$ corresponding to $\psi(x,y) = (\beta x, y)$
where $\beta$ is an order $3$ element of $\Fp$.
In this particular case, the endomorphism $\psi$ can be evaluated
using only one mulltiplication in $\Fp$.

In this work, we look for curves of small discriminant in order to
find a curve defined over the BLS12-381 scalar field where the GLV
method applies. Moreover, we are looking for secure curves in the
sense that the curve and its twist are not threaten by subgroup
attacks, meaning that the order of the curve and its quadratic twist
has a large prime factor.

As the endomorphism cost is closely related to the discriminant, we
restrict to $-D \geq -388$. Moreover, isogenous curve have the same
order and we can restrict on fundamental discriminants.

We compute an exhaustive search among all the possible discriminant
($-292 \leq -D \leq -3$). Given a discriminant $-D$, roughly half of
the curves are supersingular and hence not interesting for our
cryptographic applications. We list in
Table~\ref{tab:group-order-factorization} the ordinary curves of small
discriminant. The generation of these curves is reproducible using~\href{https://github.com/asanso/Bandersnatch/blob/main/code/small-disc-curves.py}{this
  file}.
We finally obtain an interesting curve for $-D = -8$, meaning that the
curve has an endomorphism $\psi$ satisfying $\psi^2 = [-2]$.

\begin{table*}[!ht]
    \centering\footnotesize
    \begin{tabularx}{\textwidth}{ccl}
        \toprule
                            
        $\mathbf{-D}$    & \textbf{Curve sec.}  & \textbf{Curve order} \\
        \midrule        
$-3$ & $65$-bit & $2^{2}  \cdot 3  \cdot 97  \cdot 19829809  \cdot 2514214987  \cdot 423384683867248993  \cdot p_{131}$\\
 & $14$-bit & $2^{64}  \cdot 906349^{4}  \cdot p_{28}^{4}$\\
 & $77$-bit & $7  \cdot 43  \cdot 1993  \cdot 2137  \cdot 43558993  \cdot 69032539613749  \cdot p_{154}$\\
 & $41$-bit & $3  \cdot 7  \cdot 13  \cdot 79  \cdot 2557  \cdot 33811
        \cdot 1645861201  \cdot 75881076241177 \cdot$\\
 &          & $86906511869757553  \cdot p_{82}$\\
 & $13$-bit & $3^{2}  \cdot 11^{2}  \cdot 19^{2}  \cdot 10177^{2}  \cdot 125527^{2}  \cdot 859267^{2}  \cdot 2508409^{2}  \cdot 2529403^{2}  \cdot p_{26}^{2}$\\
 & $118$-bit & $836509  \cdot p_{236}$\\
$-4$ & $59$-bit & $2^{32}  \cdot 5  \cdot 73  \cdot 906349^{2}  \cdot 254760293^{2}  \cdot p_{119}$\\
 & $37$-bit & $2^{2}  \cdot 29  \cdot 233  \cdot 34469  \cdot
        1327789373  \cdot 19609848837063073 \cdot$\\
 &          & $159032890827948314857  \cdot p_{74}$\\
 & $37$-bit & $2  \cdot 3^{2}  \cdot 11^{2}  \cdot 13  \cdot 1481  \cdot 10177^{2}  \cdot 859267^{2}  \cdot 52437899^{2}  \cdot 346160718017  \cdot p_{74}$\\
 & $57$-bit & $2  \cdot 5  \cdot 19^{2}  \cdot 1709  \cdot 125527^{2}  \cdot 2508409^{2}  \cdot 2529403^{2}  \cdot p_{114}$\\
$\mathbf{-8}$ & $\mathbf{122}$\textbf{-bit} & $\mathbf{2^{7}  \cdot 3^{3}  \cdot p_{244}}$\\
 & $\mathbf{126}$\textbf{-bit} & $\mathbf{2^{2}  \cdot p_{253}}$\\
$-11$ & $69$-bit & $5  \cdot 191  \cdot 5581  \cdot 18793  \cdot 48163  \cdot 46253594704380463613  \cdot p_{138}$\\
 & $73$-bit & $3^{3}  \cdot 11^{2}  \cdot 9269797  \cdot 17580060420191283788101  \cdot p_{147}$\\
$-19$ & $110$-bit & $7  \cdot 11^{2}  \cdot 19  \cdot 23  \cdot 397  \cdot 419  \cdot p_{220}$\\
 & $74$-bit & $3^{2}  \cdot 5  \cdot 503  \cdot 10779490483  \cdot 433275286013779991  \cdot p_{149}$\\
$-24$ & $53$-bit & $2^{2}  \cdot 3^{2}  \cdot 7  \cdot 19^{2}  \cdot 127  \cdot 29402034080953  \cdot 2970884754778276642175743  \cdot p_{106}$\\
 & $86$-bit & $2^{5}  \cdot 5  \cdot 39628279  \cdot 1626653036429383  \cdot p_{172}$\\
$-51$ & $112$-bit & $3^{2}  \cdot 5  \cdot 61223923  \cdot p_{224}$\\
 & $120$-bit & $23^{2}  \cdot 41  \cdot p_{241}$\\
$-67$ & $67$-bit & $3479887483  \cdot 56938338857  \cdot 8474085246072233  \cdot p_{135}$\\
 & $79$-bit & $3^{2}  \cdot 8478452882270519617659314063  \cdot p_{159}$\\
$-88$ & $61$-bit & $2^{2}  \cdot 11  \cdot 16984307  \cdot 24567897636186592260640293583411  \cdot p_{122}$\\
 & $66$-bit & $2^{9}  \cdot 3^{2}  \cdot 31  \cdot 6133  \cdot 116471  \cdot 69487476515565975361139  \cdot p_{133}$\\
$-132$ & $73$-bit & $2  \cdot 1753  \cdot 101235113104036296384208928969  \cdot p_{147}$\\
 & $92$-bit & $2  \cdot 3^{2}  \cdot 7^{2}  \cdot 11  \cdot 23  \cdot 587  \cdot 701  \cdot 32299799971  \cdot p_{184}$\\
$-136$ & $62$-bit & $2^{3}  \cdot 7^{3}  \cdot 19^{3}  \cdot 10939  \cdot 11131315086725327441688173207  \cdot p_{125}$\\
 & $87$-bit & $2^{2}  \cdot 5  \cdot 5741  \cdot 30851  \cdot 533874022134253  \cdot p_{175}$\\
$-228$ & $114$-bit & $2  \cdot 3^{2}  \cdot 19  \cdot 89  \cdot 5189  \cdot p_{228}$\\
 & $81$-bit & $2  \cdot 947  \cdot 277603  \cdot 28469787063396608749  \cdot p_{162}$\\
$-244$ & $89$-bit & $2^{2}  \cdot 13  \cdot 523  \cdot 1702319  \cdot 2827715661581  \cdot p_{179}$\\
 & $88$-bit & $2^{8}  \cdot 3^{2}  \cdot 5  \cdot 71  \cdot 907  \cdot 2749  \cdot 146221  \cdot 2246269  \cdot p_{176}$\\
$-264$ & $83$-bit & $2^{3}  \cdot 11  \cdot 131  \cdot 12543757399  \cdot 2818746796297  \cdot p_{167}$\\
 & $82$-bit & $2^{2}  \cdot 3  \cdot 5^{2}  \cdot 2287  \cdot 2134790941497418864559  \cdot p_{165}$\\
$-276$ & $70$-bit & $2  \cdot 11^{2}  \cdot 8839  \cdot 78797899  \cdot 323360863688748558301  \cdot p_{140}$\\
 & $88$-bit & $2  \cdot 3  \cdot 5  \cdot 6197  \cdot 138617  \cdot 16664750312513  \cdot p_{177}$\\
$-292$ & $92$-bit & $2  \cdot 54983  \cdot 5220799  \cdot 2671917733  \cdot p_{185}$\\
 & $86$-bit & $2  \cdot 11^{2}  \cdot 149  \cdot 354689  \cdot
24012883  \cdot 32483123  \cdot p_{172}$\\
\bottomrule
    \end{tabularx}
    \caption{Curves for discriminants $-3 \geq -D \geq -292$.}
    \label{tab:group-order-factorization}
\end{table*}

In the next section, we present this curve and several properties
related to the  GLV scalar multiplication in this case.

\SM{TODO: explain our algorithm for finding curves, subgroup and twist
  security.}

\SM{TODO: not defined: quadratic twist, isogenous curves, fundamental
  discriminant.}

\section{Bandersnatch}\label{sec:bandersnatch}

The Bandersnatch is obtained from a discriminant $-D = -8$, meaning
that its $j$-invariant is a root of the Hilbert class polynomial
$H_{-8}(X) = X-8000$.
From the $j$-invariant, it is easy to obtain the equation of
Bandersnatch (or its quadratic twist).
We provide in this section several representations of Bandersnatch.

\paragraph{Security of Bandersnatch.}
The bandersnatch curve order is $2^2\cdot r$ for a $253$-bit long
prime $r$.
% 13108968793781547619861935127046491459309155893440570251786403306729687672801
The quadratic twist of $E_\text{TE}$ has order
$2^7 \cdot 3^3 \cdot r'$, where $r'$ is another prime of $244$ bits.
%15172417585395309745210573063711216967055694857434315578142854216712503379
Hence, the Bandersnatch curve satisfies twist security after a quick cofactor
check.
The different attacks on the Bandersnatch curve subgroups lead to $125.75$ bits
of security.

\subsection{Weierstrass curve}
The Bandersnatch curve can be represented in the Weierstrass model
using the following parameters:
\begin{verbatim}
p=0x73eda753299d7d483339d80809a1d80553bda402fffe5bfeffffffff00000001
E=EllipticCurve(GF(p), [-3763200000, -78675968000000])
\end{verbatim}

\begin{remark}
  Using SageMath, \texttt{EllipticCurve\_from\_j(8000)} computes a
  Weierstrass equation of the quadratic twist of Bandersnatch.
  Bandersnatch using
  \texttt{EllipticCurve\_from\_j(8000).quadratic\_twist()}.
\end{remark}

\subsection{Twisted Edwards curve}
It can also be represented in Twisted Edwards coordinates.
In this model, the Bandersnatch curve can be defined by the equation
$$E_\text{TE}:-5x^2+y^2 = 1 + \frac{138827208126141220649022263972958607803}{171449701953573178309673572579671231137}x^2y^2$$ 

From that, we exhibit the degree $2$ endomorphism in Twisted Edwards
coordinates (where $a_i, b_i, c_i \in \Fp$):
$$\psi(x,y,z) = (xa_1(y+a_2z)(y+a_3z), b_1(y+b_2z)(y+b_3z)yz^2,
(y+c_1z)(y+c_2z)yz^2).$$
This map can be computed in 17 multiplications and 6 additions modulo $p$.
A twisted Edwards curve is always birationally equivalent to a
Montgomery curve.

\subsection{Montgomery curve}
While the Twisted Edwards model fits better for $\mathbb F_p$ circuit
arithmetic, we provide here the Montgomery version because the scalar
multiplication is more efficient in this context.
$$E_\text{M}: By^2 = x^3 + Ax^2 + x$$
\begin{align*}
  B &= \text{\small{\tt 0x300c3385d13bedb7c9e229e185c4ce8b1dd3b71366bb97c30855c0aa41d62727}}\\
  A &= \text{\small{\tt 0x4247698f4e32ad45a293959b4ca17afa4a2d2317e4c6ce5023e1fd63d1b5de98}}.
\end{align*}


\section{Comparison}\label{sec:comparison}

\SM{TODO: talk about $a=-1$ vs $a=-5$, and give more precise complexity
analysis and benchmarks.}

%### Scalar multiplication improvement
From the efficient endomorphism $\psi$, it is easy to apply the GLV method and improve the scalar multiplication cost:
Roughly, a scalar multiplication $[n]P$ cost $(\log n) \text{Dbl} + (\log n/2) \text{Add}$.
Using the GLV endomorphism, we can compute $[n]P$ using $(\log n/2 )\text{Dbl} + (3\log n/8) \text{Add}$, plus few precomputations.

We performed `python` benchmarks between the double-and-add algorithm and the GLV method applied in the case of our curve, and the GLV version is about 30\% faster 
\footnote{Source code \url{https://github.com/asanso/Bandersnatch/}.}

\SM{Rust benchmarks will be available soon!}

\bigskip
\paragraph*{\textbf{Acknowledgments.}} we would like to thank Luca De Feo, Justin Drake, Dankrad Feist, Daira Hopwood and Zhenfei Zhang for fruitful discussions.

\bibliography{bandersnatch,cryptobib/abbrev3,cryptobib/crypto}
\bibliographystyle{unsrt}
\end{document}
