\documentclass[sigconf,nonacm]{acmart}

\usepackage{hyperref}
\usepackage[capitalize]{cleveref}
\usepackage{tikz}
\usepackage[all]{xy}
\usepackage[foot]{amsaddr}
\usepackage{amsmath,amsfonts,amsthm}
\usepackage{xcolor}
\usepackage{tabularx}
\usepackage{booktabs}

\newcommand{\SM}[1]{\color{blue}#1\color{black}}
\newcommand{\ZZ}[1]{\color{cyan}#1\color{black}}
\newcommand{\AS}[1]{\color{red}#1\color{black}}

\newcommand{\Q}{\ensuremath{\mathbb Q}}
\newcommand{\Z}{\ensuremath{\mathbb Z}}
\newcommand{\Fp}{\ensuremath{\mathbb F_p}}
\newcommand{\End}{\ensuremath{\text{End}}}

\usepackage{listings}% http://ctan.org/pkg/listings
\lstset{
  basicstyle=\ttfamily,
  mathescape
}

\theoremstyle{definition}
\newtheorem*{remark}{Remark}

\title{Bandersnatch: a fast elliptic curve built over the BLS12-381
  scalar field}
\author{No author given}
\date{}

\makeatletter
\newcommand{\verbatimfont}[1]{\renewcommand{\verbatim@font}{\ttfamily#1}}
\makeatother

\begin{document}

\verbatimfont{\small}%

% \medskip
\begin{abstract}
 In this paper, we introduce Bandersnatch, a new elliptic curve
 built over the BLS12-381 scalar field. The curve is equipped with an efficient
 endomorphism, allowing a fast scalar multiplication algorithm.
 Our benchmark shows that the multiplication is 42\% faster, 
 21\% reduction in terms of circuit size in the 
 form of rank 1 constraint systems (R1CS), 
 and 10\% reduction in terms of plonk circuit,
 compared to another curve, called Jubjub, having similar
 properties. Many zero-knowledge proof systems that rely on
 the Jubjub curve can benefit
 from our result.
\end{abstract}

\maketitle

\section{Introduction}
%BLS12-381
BLS12-381 is a pairing-friendly curve discovered by Sean
Bowe \cite{bls12381} in 2017.
Currently, BLS12-381 is undergoing a standardization process 
from the
IRTF Crypto Forum Research Group, and is
universally used for digital
signatures and zero-knowledge proofs by many projects orbiting in the
blockchain universe: Zcash, Ethereum 2.0, \href{https://anoma.network}{Anoma}, Skale, Algorand, Dfinity,
Chia, and more.
%jubjub
The ZCash team
introduced Jubjub \cite{jubjub}, an
elliptic curve built over the BLS12-381 scalar field $\mathbb
F_{r_\text{BLS}}$.
This curve is not pairing-friendly, but leads to constructions where
$\mathbb F_{r_\text{BLS}}$ arithmetic circuits can be manipulated
using the BLS12-381 curve.
The Jubjub curve can be represented in the twisted Edwards
coordinates, allowing efficiency inside zk-SNARK circuits.
In order for some cryptographic applications to scale, it is necessary to
have efficient scalar multiplication on the non-pairing-friendly
curve.
The main drawback of Jubjub is the slow scalar multiplication
algorithm compared, for example, with the ``Bitcoin curve''
(SECP256k1).
It comes from the fact that the curve does not have an efficiently
computable endomorphism, necessary for computing scalar
multiplications using the GLV method~\cite{C:GalLamVan01} (a technique
protected by a US patent until Sep 2020~\cite{glvpatent}, but that 
expired and is
freely usable now).

\paragraph{Our contribution.}
The Jubjub curve is a curve with a large discriminant, meaning that
the GLV method is not possible on this curve.
We performed an exhaustive search of curves of small discriminant,
defined over the BLS12-381 scalar field. This way, we obtain an
elliptic curve using the Complex Multiplication
method~\cite{MC:AtkMor93}, where the scalar multiplication algorithm
is efficient thanks to the GLV method~\cite{C:GalLamVan01}.

We implement this curve in Rust, using
the {Arkworks framework} \cite{arkworks}, and release our
code to the open domain~\cite{bandersnatch-rust}.
Table~\ref{tab:comp} shows a comparison of Bandersnatch curve
and Jubjub curve. 
Details deferred to Section~\ref{sec:comparison}.

\begin{table*}[ht] %\small
  \centering
  
  \begin{tabular}{|l|c|c|c|}\hline
      & multiplication cost & R1CS constraints & TurboPlonk constraints  \\\hline\hline
    Jubjub & 75 $\mu$s  & 3177 &  1865\\\hline
    Bandersnatch & 44 $\mu$s  & 2621 & 1669\\\hline\hline   
    Improvement & 42\% & 21\% & 10\%\\\hline
  \end{tabular}
  \caption{Bandersnatch vs Jubjub}
  \label{tab:comp}
\end{table*}

To demonstrate how a zero-knowledge proof system can benefit
form our curve, we also report the number of 
% , in terms of both group multiplications and 
% the number of 
constraints one needs to express a group multiplication
in the forms of 
two commonly used circuit descriptions for zero-knowledge 
proof systems, namely,
rank one constraint system (R1CS), 
and the plonk circuit.
A group multiplication takes 
2621 and 1669
constraints, respectively, when the point is in affine form over the 
twisted Edwards curve,
%, and 
%2361
%constraints when the point is in the projective form
%over the short Weierstrass from.
%Both figures 
% This matches what we have for Jubjub curve.
yielding a 21\% and 10\% improvement over the Jubjub curve.

\paragraph{Organization of the paper.}
In Section~\ref{sec:small-disc-curves}, we describe how we obtained
several curves allowing the GLV method together with cryptographic
security.
Then, we introduce in Section~\ref{sec:bandersnatch} the Bandersnatch
curve in different models (in Weierstrass, Montgomery and twisted Edwards
coordinates).
Finally, we compare the scalar multiplication algorithm over
the Bandersnatch and the Jubjub curves in
Section~\ref{sec:comparison} from a practical point of view.


\section{Small discriminant curves}\label{sec:small-disc-curves}

The GLV method~\cite{C:GalLamVan01} is a well known trick for accelerating
scalar multiplication over particular curve. In a nutshell, it applies
to elliptic curves where an endomorphism $\psi$ can be efficiently computed.
The GLV method applies in particular for $j$-invariant $j=0$
(resp. $j=1728$) curves because a non-trivial automorphism can be
computed using only one modular multiplication. % (see Example~\ref{ex:psi-j0}).
The method also applies for other curves where the endomorphism is
slightly more expensive, called \emph{small discriminant} curves.

% notations
Let $E$ be an elliptic curve defined over $\Fp$ of trace $t$. $E$ and
its quadratic twist $E^t$ are $\mathbb F_{p^2}$-isomorphic curves and
their orders over $\Fp$ are closely related with the trace
$t$:
$$\#E(\Fp) = p+1-t\qquad \#E^t(\Fp) = p+1+t.$$
See~\cite{Silverman86} for a complete introduction to elliptic curves.
In this work, we are looking for cryptographic applications based on
ordinary elliptic curves, meaning that we look for $t\not\equiv 0
\bmod p$. The endomorphism ring of these curves have a particular
structure: $\End(E)$ is an order of the imaginary quadratic field
$\Q(\sqrt{t^2-4p})$.
From now, we denote $-D$ to be the discriminant of $\End(E)$, and
$\{\text{Id},\psi\}$ a basis of the endomorphism ring.
The fundamental discriminant corresponds to the discriminant of the
maximal order containing $\End(E)$.
This way, $\psi$ is of degree $\frac{D+1}4$ or $D/4$ depending on the
value of $D$ modulo $4$, and $\psi$ can be defined using polynomials
of degree $O(D)$ thanks to the Vélu's formulas~\cite{velu71}.
Thus, the evaluation of $\psi$ is efficient only for curves of small
discriminant.

In this work, we restrict to curves defined over the BLS12-381 scalar
field $\mathbb F_{r_\text{BLS}}$. From now, we denote $p=r_\text{BLS}$
and we look for curves with a $128$-bit cryptographic security.
Curves with $-D=-3$ and $-4$ do not have a large prime subgroup
defined over $\Fp$.
Hence, we look for small discriminant $-D<-4$ curves with subgroup and
twist-subgroup security. It means that $\#E(\Fp)$ has a roughly 256-bit
prime factor, as well as $\#E^t(\Fp)$.

As the endomorphism cost is closely related to the discriminant, we
restrict to $-D \geq -388$ so that $\psi$ can be efficiently computed.
Moreover, we restrict on fundamental discriminants (discriminants
of the maximal orders of imaginary quadratic fields). We denote
$\mathcal O_{-D}$ the maximal order of discriminant $-D$. Elliptic
curves with $\End(E) \subset \mathcal O_{-D}$ are isogenous curves,
meaning that there is a rational map between them. Isogenous curves
have the same order so that we can restrict on fundamental
discriminants for our search.

We compute an exhaustive search among all the possible (fundamental)
discriminants ($-292 \leq -D \leq -3$).
Given a discriminant $-D$, roughly half of the curves are
supersingular and hence not relevant to our cryptographic
applications.
We list in Table~\ref{tab:group-order-factorization} the ordinary
curves we obtained. In this table, $p_n$ denotes a prime of $n$ bits.
The generation of these curves is reproducible
using~\href{https://github.com/asanso/Bandersnatch/blob/main/python-ref-impl/small-disc-curves.py}{this
  file}.
We finally obtain an interesting curve for $-D = -8$ with large prime
order subgroups on both the curve and its twist.
We present in Section~\ref{sec:bandersnatch} the curve in several
models, together with the endomorphism in order to apply the GLV
scalar multiplication algorithm.

\begin{table*}[!ht]
    \centering%\footnotesize
    \begin{tabularx}{\textwidth}{ccl}
        \toprule                            
        $\mathbf{-D}$    & \textbf{Curve sec.}  & \textbf{Curve order} \\
        \midrule        
$-3$ & $65$-bit & $2^{2}  \cdot 3  \cdot 97  \cdot 19829809  \cdot 2514214987  \cdot 423384683867248993  \cdot p_{131}$\\
 & $14$-bit & $2^{64}  \cdot 906349^{4}  \cdot p_{28}^{4}$\\
 & $77$-bit & $7  \cdot 43  \cdot 1993  \cdot 2137  \cdot 43558993  \cdot 69032539613749  \cdot p_{154}$\\
 & $41$-bit & $3  \cdot 7  \cdot 13  \cdot 79  \cdot 2557  \cdot 33811
        \cdot 1645861201  \cdot 75881076241177 \cdot$\\
 &          & $86906511869757553  \cdot p_{82}$\\
 & $13$-bit & $3^{2}  \cdot 11^{2}  \cdot 19^{2}  \cdot 10177^{2}  \cdot 125527^{2}  \cdot 859267^{2}  \cdot 2508409^{2}  \cdot 2529403^{2}  \cdot p_{26}^{2}$\\
 & $118$-bit & $836509  \cdot p_{236}$\\
$-4$ & $59$-bit & $2^{32}  \cdot 5  \cdot 73  \cdot 906349^{2}  \cdot 254760293^{2}  \cdot p_{119}$\\
 & $37$-bit & $2^{2}  \cdot 29  \cdot 233  \cdot 34469  \cdot
        1327789373  \cdot 19609848837063073 \cdot$\\
 &          & $159032890827948314857  \cdot p_{74}$\\
 & $37$-bit & $2  \cdot 3^{2}  \cdot 11^{2}  \cdot 13  \cdot 1481  \cdot 10177^{2}  \cdot 859267^{2}  \cdot 52437899^{2}  \cdot 346160718017  \cdot p_{74}$\\
 & $57$-bit & $2  \cdot 5  \cdot 19^{2}  \cdot 1709  \cdot 125527^{2}  \cdot 2508409^{2}  \cdot 2529403^{2}  \cdot p_{114}$\\
$\mathbf{-8}$ & $\mathbf{122}$\textbf{-bit} & $\mathbf{2^{7}  \cdot 3^{3}  \cdot p_{244}}$\\
 & $\mathbf{126}$\textbf{-bit} & $\mathbf{2^{2}  \cdot p_{253}}$\\
$-11$ & $69$-bit & $5  \cdot 191  \cdot 5581  \cdot 18793  \cdot 48163  \cdot 46253594704380463613  \cdot p_{138}$\\
 & $73$-bit & $3^{3}  \cdot 11^{2}  \cdot 9269797  \cdot 17580060420191283788101  \cdot p_{147}$\\
$-19$ & $110$-bit & $7  \cdot 11^{2}  \cdot 19  \cdot 23  \cdot 397  \cdot 419  \cdot p_{220}$\\
 & $74$-bit & $3^{2}  \cdot 5  \cdot 503  \cdot 10779490483  \cdot 433275286013779991  \cdot p_{149}$\\
$-24$ & $53$-bit & $2^{2}  \cdot 3^{2}  \cdot 7  \cdot 19^{2}  \cdot 127  \cdot 29402034080953  \cdot 2970884754778276642175743  \cdot p_{106}$\\
 & $86$-bit & $2^{5}  \cdot 5  \cdot 39628279  \cdot 1626653036429383  \cdot p_{172}$\\
$-51$ & $112$-bit & $3^{2}  \cdot 5  \cdot 61223923  \cdot p_{224}$\\
 & $120$-bit & $23^{2}  \cdot 41  \cdot p_{241}$\\
$-67$ & $67$-bit & $3479887483  \cdot 56938338857  \cdot 8474085246072233  \cdot p_{135}$\\
 & $79$-bit & $3^{2}  \cdot 8478452882270519617659314063  \cdot p_{159}$\\
$-88$ & $61$-bit & $2^{2}  \cdot 11  \cdot 16984307  \cdot 24567897636186592260640293583411  \cdot p_{122}$\\
 & $66$-bit & $2^{9}  \cdot 3^{2}  \cdot 31  \cdot 6133  \cdot 116471  \cdot 69487476515565975361139  \cdot p_{133}$\\
$-132$ & $73$-bit & $2  \cdot 1753  \cdot 101235113104036296384208928969  \cdot p_{147}$\\
 & $92$-bit & $2  \cdot 3^{2}  \cdot 7^{2}  \cdot 11  \cdot 23  \cdot 587  \cdot 701  \cdot 32299799971  \cdot p_{184}$\\
$-136$ & $62$-bit & $2^{3}  \cdot 7^{3}  \cdot 19^{3}  \cdot 10939  \cdot 11131315086725327441688173207  \cdot p_{125}$\\
 & $87$-bit & $2^{2}  \cdot 5  \cdot 5741  \cdot 30851  \cdot 533874022134253  \cdot p_{175}$\\
$-228$ & $114$-bit & $2  \cdot 3^{2}  \cdot 19  \cdot 89  \cdot 5189  \cdot p_{228}$\\
 & $81$-bit & $2  \cdot 947  \cdot 277603  \cdot 28469787063396608749  \cdot p_{162}$\\
$-244$ & $89$-bit & $2^{2}  \cdot 13  \cdot 523  \cdot 1702319  \cdot 2827715661581  \cdot p_{179}$\\
 & $88$-bit & $2^{8}  \cdot 3^{2}  \cdot 5  \cdot 71  \cdot 907  \cdot 2749  \cdot 146221  \cdot 2246269  \cdot p_{176}$\\
$-264$ & $83$-bit & $2^{3}  \cdot 11  \cdot 131  \cdot 12543757399  \cdot 2818746796297  \cdot p_{167}$\\
 & $82$-bit & $2^{2}  \cdot 3  \cdot 5^{2}  \cdot 2287  \cdot 2134790941497418864559  \cdot p_{165}$\\
$-276$ & $70$-bit & $2  \cdot 11^{2}  \cdot 8839  \cdot 78797899  \cdot 323360863688748558301  \cdot p_{140}$\\
 & $88$-bit & $2  \cdot 3  \cdot 5  \cdot 6197  \cdot 138617  \cdot 16664750312513  \cdot p_{177}$\\
$-292$ & $92$-bit & $2  \cdot 54983  \cdot 5220799  \cdot 2671917733  \cdot p_{185}$\\
 & $86$-bit & $2  \cdot 11^{2}  \cdot 149  \cdot 354689  \cdot
24012883  \cdot 32483123  \cdot p_{172}$\\
\bottomrule
    \end{tabularx}
    \caption{Curves for discriminants $-3 \geq -D \geq -292$.}
    \label{tab:group-order-factorization}
\end{table*}

\section{Bandersnatch}\label{sec:bandersnatch}

The Bandersnatch is obtained from a discriminant $-D = -8$, meaning
that the endomorphism ring is $\Z[\sqrt{-2}]$.
We obtain the curve $j$-invariant using the Complex Multiplication
method, based on the Hilbert class polynomial $H_{-D}(X)$.
The roots of $H_{-D}$ are $j$-invariants of elliptic curves whose
endomorphism ring is of discriminant $-D$.
From a $j$-invariant, we obtain the curve equation in different
models. Before looking into the details of three different representations,
we briefly recall how to exhibit the degree $2$ endomorphism $\psi$.

\paragraph{Degree 2 endomorphism.}
The endomorphism $\psi$ has a kernel generated by a $2$-torsion
point. Hence, we can obtain the rational maps defining $\psi$ by
looking at the curves $2$-isogenous to Bandersnatch. Only one has the
same $j$-invariant, meaning that up to an isomorphism, the Vélu's
formulas~\cite{velu71} let us obtain compute $\psi$.
For cryptographic use-cases, we are interested in computing $\psi$ on
the $p_{253}$-order subgroup of the curve. On these points, $\psi$
acts as a scalar multiplication by the eigenvalue
% \begin{lstlisting}
%   $\lambda$ = 0x13b4f3dc4a39a493edf849562b38c72b
%                 cfc49db970a5056ed13d21408783df05
% \end{lstlisting}
\begin{align*}
  \lambda = \text{\small{\tt
    0x13b4f3dc4a39a493edf849562b38c72b}}&\\
    \text{\small{\tt  cfc49db970a5056ed13d21408783df05}}&
\end{align*}
By construction, $\psi$ is the endomorphism $\sqrt{-2}\in \mathcal
O_{-8}$. Thus, $\lambda$ satisfies
$\lambda^2+2 = 0 \bmod p_{253}$.
In the following sections, we provide details on the curve equation,
the $\psi$ rational maps, and a generator of the $p_{253}$-order
subgroup in the case of the affine Weierstrass, projective Montgomery
and projective twisted Edwards representations.
The parameters are reproducible using the script
of~\href{https://github.com/asanso/Bandersnatch/blob/main/python-ref-impl/get\_params.py}{this
  file}.

\subsection{Weierstrass curve}\label{sec-w-curve}
\paragraph{Curve equation.}
The Bandersnatch curve can be represented in the Weierstrass model
using the equation
$$E_W:y^2 = x^3 -3763200000x -78675968000000.$$
%% \begin{verbatim}
%% p=0x73eda753299d7d483339d80809a1d80553bda402fffe5bfeffffffff00000001
%% E=EllipticCurve(GF(p), [-3763200000, -78675968000000])
%% \end{verbatim}

\paragraph{Endomorphism.}
The endomorphism $\psi$ can obtained using the method detailed above.
We obtain the following expression:
\begin{align*}
  \psi_\text{W}&(x,y) = \\
  &\left(u^2\cdot \frac{x^2+44800x+2257920000}{x+44800}, u^3\cdot
y\cdot \frac{x^2+2\cdot 44800x+t_0}{(x+44800)^2}\right),
\end{align*}
where
\begin{verbatim}
        u  = 0x50281ac0f92fc1b20fd897a16bf2b9e1
               32bdcb06721c589296cf82245cf9382d,
        t0 = 0x73eda753299d7d483339d80809a1d805
               53bda402fffe5bfefffffffef10be001.
\end{verbatim}

\paragraph{Subgroup generator.}
The generator of the $p_{253}$-order subgroup is computed by finding the
lexicographically smallest valid $x$-coordinate of a point of the
curve, and scaling it by the cofactor $4$ such that the result is not
the point at infinity. From a point with $x=2$, we obtain a generator $E_W(x_W,y_W)$ where:
\begin{verbatim}
        xW = 0x0a76451786f95a802c0982bbd0abd68e
               41b92adc86c8859b4f44679b21658710,
        yW = 0x44d150c8b4bd14f79720d021a839e7b7
               eb4ee43844b30243126a72ac2375490a.
\end{verbatim}


\subsection{Twisted Edwards curve}
\paragraph{Curve equation.}
Bandersnatch can also be represented in twisted Edwards coordinates,
where the group law is complete.
In this model, the Bandersnatch curve can be defined by the equation
\begin{align*}
&E_\text{TE}:-5x^2+y^2 = 1 + dx^2y^2, \\
&d=\frac{138827208126141220649022263972958607803}{171449701953573178309673572579671231137}.
\end{align*}

Twisted Edwards group law is more efficient with a coefficient
$a = -1$ (see~\cite{AC:HWCD08} for details).
In our case, $-5$ is not a square in $\Fp$. Thus, the curve with
equation $-x^2+y^2 = 1 -dx^2y^2/5$ is the quadratic twist of
Bandersnatch. We provide a representation with $a=-5$, leading to a
slightly more expensive group law because multiplying by $-5$ is more
expensive than a multiplication by $-1$, but this cost will be
neglected compared to the improvement of the GLV method. See
Section~\ref{sec:comparison} for details.

\paragraph{Endomorphism.}
From this representation, we exhibit the degree $2$ endomorphism in
twisted Edwards coordinates:
$$
\psi_\text{TE}(x,y,z) = \left(f(y)h(y), g(y)xy, h(y)xy\right)$$
with
\begin{align*}
  f(y) &= c(z^2-y^2),\\
  g(y) &= b(y^2+bz^2),\\
  h(y) &= y^2-bz^2,
\end{align*}
and
\begin{verbatim}
        b = 0x52c9f28b828426a561f00d3a63511a88
              2ea712770d9af4d6ee0f014d172510b4,
        c = 0x6cc624cf865457c3a97c6efd6c17d107
              8456abcfff36f4e9515c806cdf650b3d.
\end{verbatim}
This map can be computed in 3 additions and 9 multiplications by
first computing $xy$, $y^2$, $z^2$ and $bz^2$.

\paragraph{Subgroup generator.}
The generator of the $p_{253}$-order subgroup obtained in
Section~\ref{sec-w-curve} has twisted coordinates
of the form $E_\text{TE}(x_\text{TE},y_\text{TE},1)$ with
$$x_{TE} = \text{{\small{\tt 0x29c132cc2c0b34c5743711777bbe42f32b79c022ad998465e1e71866a252ae18,}}}$$
$$y_{TE} = \text{{\small{\tt 0x2a6c669eda123e0f157d8b50badcd586358cad81eee464605e3167b6cc974166.
}}}$$


\subsection{Montgomery curve}
\paragraph{Curve equation.}
A twisted Edwards curve is always birationally equivalent to a
Montgomery curve. We obtain the mapping between these two
representations following~\cite{JCEng:CosSmi18}.
While the twisted Edwards model fits better for $\mathbb F_p$ circuit
arithmetic and more generally for the zero-knowledge proof context, we
provide here the Montgomery version because the scalar multiplication
is more efficient in this model:
$$E_M: By^2 = x^3 + Ax^2 + x$$
\begin{verbatim}
        B = 0x300c3385d13bedb7c9e229e185c4ce8b
              1dd3b71366bb97c30855c0aa41d62727,
        A = 0x4247698f4e32ad45a293959b4ca17afa
              4a2d2317e4c6ce5023e1fd63d1b5de98.
\end{verbatim}

\paragraph{Endomorphism.}
Montgomery curves allow efficient scalar multiplication using the
Montgomery ladder. We provide here the endomorphism $\psi$ in this
model:
$$\psi_\text{M}(x,-,z) = (-(x-z)^2 - cxz, -, 2xz)$$
with
\begin{verbatim}
        c = 0x4247698f4e32ad45a293959b4ca17afa
              4a2d2317e4c6ce5023e1fd63d1b5de9a.
\end{verbatim}

\paragraph{Subgroup generator.}
The generator of the $p_{253}$-order subgroup given above is of the
form $E_M(x_M,-,1)$ with:
\begin{verbatim}
        xM = 0x67c5b5fed18254e8acb66c1e38f33ee0
               975ae6876f9c5266a883f4604024b3b8.
\end{verbatim}


\subsection{Security of Bandersnatch}

The Bandersnatch curve order is $2^2\cdot r$ for a $253$-bit long
prime $r$.
% 13108968793781547619861935127046491459309155893440570251786403306729687672801
Its quadratic twist has order
$2^7 \cdot 3^3 \cdot r'$, where $r'$ is another prime of $244$ bits.
%15172417585395309745210573063711216967055694857434315578142854216712503379
Hence, the Bandersnatch curve satisfies twist security after a quick cofactor
check.
We estimate that the Bandersnatch curve (resp. its quadratic twist)
has $126$ bits of security (resp. $122$ bits of security).

% \ZZ{TODOs: 1. expand the security section 2. hash to curve (not essential, but nice to have)
% 3. serialization (required by a spec rather than an academic paper. may be borrow 
% some ed25519 trick where the cofactor is 8?)}

\section{Comparison}\label{sec:comparison}

%### Scalar multiplication improvement
The twisted Edwards representation is mostly used in practice, and we
now focus on the comparison between Jubjub and Bandersnatch in this
model.

\subsection{Scalar multiplications for a variable base point}
% Jubjub scalar multiplication
Because of its large discriminant, the scalar multiplication on Jubjub
is a basic double-and-add algorithm, meaning that it computes a
multiplication by $n$ in $\log n$ doublings and $\log n/2$
additions (in average) on the curve. 

% Bandersnatch scalar multiplication
The endomorphism $\psi$ lets us compute the scalar multiplication
faster than a double-and-add algorithm with few precomputations. For a
point $P$ and a scalar $n$, we first evaluate $\psi$ at $P$ and
decompose $n = n_1 + \lambda n_2$. Then a multi scalar multiplication
is computed in $\log n/2$ doublings and $3\log n/8$ additions (in average) on the curve.

% Benchmarks
We benchmarked our implementation with both GLV enabled and disabled, and 
compared it with Arkworks' own Jubjub implementation. 
Our benchmark is conducted over an AMD 5900x CPU, with Ubuntu 20.04,
rust 1.52 stable version, and Arkwork 0.3.0 release version.
We used~\href{https://docs.rs/criterion}{\texttt{criterion}
  micro-benchmark toolchain},  version 0.3.0, for data collection. We
compile Arkworks with two options, namely \texttt{default} and
\texttt{asm}, respectively.
The \texttt{default} setup relies on \texttt{num\_bigint} crate for
large integer arithmetics, while \texttt{asm} turns on assembly for
finite field multiplication. 

Arkworks use the aforementioned double-and-add
multiplication methodology, without side channel protections such 
as Montgomery ladders. Our non-GLV implementation also follows
this design. For our GLV implementation, there are three components,
namely, the endomorphism, the scalar decomposition, and the
multi scalar multiplication (MSM). We implement those schemes and 
present the micro-benchmarks in Table~\ref{tab:comp_full}.
Specifically, we do not use the MSM implementation in Arkworks:
our scalars, after the decomposition, contain roughly 128 bits
of leading zeros, and our own MSM implementation is 
optimized for this setting.

Table~\ref{tab:comp_full} presents the full picture of the benchmark.
When GLV is disabled, we observe a similar but a little worse 
performance for Bandersnatch curve, compared to
the Jubjub curve, due to the slightly larger coefficient 
$a=-5$ and a larger scalar field of 253 bits (Jubjub curve has $a=-1$
and a scalar field of 252 bits).
When GLV is enabled, we report a 45\% improvement of the Bandersnatch
curve, and a 42\% improvement over the Jubjub curve.

\begin{table}[ht] %\small
  \centering
  
  \begin{tabular}{|l|c|c|}\hline
      & \texttt{default} & \texttt{asm}\\\hline\hline
    Jubjub & 75 $\mu$s & 69 $\mu$s \\\hline\hline
    Bandersnatch without GLV & 78 $\mu$s & 72 $\mu$s  \\\hline\hline   
    Bandersnatch with GLV& 44 $\mu$s & 42 $\mu$s \\\hline
    \ \ \ Endomorphism & 2.4 $\mu$s& 1.8 $\mu$s\\\hline
    \ \ \ Scalar decomposition & 0.75 $\mu$s & 0.7 $\mu$s \\\hline
    \ \ \ multi scalar multiplication & 42 $\mu$s &  40.8 $\mu$s\\\hline\hline
    Overall Improvement & 42\% & 39\% \\\hline
  \end{tabular}
  \caption{Bandersnatch vs Jubjub: Performance}
  \label{tab:comp_full}
\end{table}

To make a meaningful comparison, we benchmark
the cost of the group multiplication over the default generators.
Note that Arkworks do not implement optimizations for 
fixed generators nonetheless. We then sample field elements 
uniformly at random, for each new iteration, and the benchmark
result is consolidated over 100 iterations.

\subsection{Multi scalar multiplications}
This section reports the performance of variable base 
multi scalar multiplications (MSM). Note that this MSM is 
compatible, but
different
from the MSM inside the GLV. 
In particular, for a sum of $k$ scalar multiplications,
we report the data point for: 
\begin{itemize}
  \item invoking the MSM over the $k$ base scalars randomly sampled,
    expected to be around 256 bits;
  \item using GLV endomorphism to break the $k$ base scalars into $2k$
    new base scalars, of halved size, i.e.~of 128 bits.
\end{itemize}
The data is presented in Figure~\ref{fig:msm}. Specifically, 
as a baseline, the trivial solution, captained by 
{\em GLV without 
MSM}, is the product of the number of bases and the cost of
doing a single GLV multiplication. 
Note that the MSM algorithms incur an overhead to build some
tables, which make them less favorable compared to the trivial
solution when dimension is really small. For a dimension greater 
than 4, MSM algorithms begin to out-perform trivial solutions.
For dimension greater than 128, it is more efficient to 
invoke the MSM directly, rather than doing it over the GLV basis.
The reason is that the size of the basis becomes too large, so
that the gain we get from halving the scalars is offset from the
gain we get from halving the basis. We remark that this threshold
point is platform dependent.
\begin{figure*}[h]\label{fig:msm}\centering 
  \includegraphics[width=12cm]{fig/msm.pdf}
\end{figure*}

\subsection{SNARK constraints}
The Bandersnatch curve is zk-SNARK friendly: its 
base field matches the scalar field for the BLS12-381 curve, a 
pairing-friendly curve, on top of which people build zk-SNARK
systems, such as Groth16 \cite{EC:Groth16} or Plonk \cite{EPRINT:GabWilCio19}.
In such a setting, the prover can sufficiently argue certain 
relationships over arithmetic circuits rather than binary
circuits.

There are two common ways to express operations in SNARK-friendly circuits, namely
\begin{itemize}
  \item Rank-1 constraint system
  (R1CS)
  \item Plonk circuit.
\end{itemize}
% The circuit is expressed in a form of Rank-1 constraint system
% (R1CS), and i
In general, the complexity is determined by the 
number of constraints for both cases.
The R1CS is universal, in that it is compatible with multiple 
prover front-end. The plonk circuit is usually bounded to a specific
plonk proving system, since Plonk prove system allows for customized
selectors, and thus, customized circuits descriptions. For plonk
circuit,
our analysis is based on an ECC optimized plonk system, with 
efficient range proofs from dynamic lookup tables \cite{EPRINT:GabWil20}.


% We therefore report the number of constraints to perform
% a group operation on both Jubjub curve and Bandersnatch 
% curve. We observe that it takes 3177 constraints for the 
% Jubjub curve, for which the statement is argued over 
% the Edwards affine form coordinates. 
Precisely, we list the breakdown numbers in Table~\ref{tab:r1cs_full};
the  Bandersnatch with GLV constraints count is a little higher than
the sum of its components due to some overhead during setup phase.

\begin{table*}[h] %\small
  \centering  
  \begin{tabular}{|l|c|c|}\hline
    & \texttt{R1CS Constraints}     & \texttt{Plonk Constraints}\\\hline\hline
    Jubjub& 3177 & 1865\\\hline
    Bandersnatch without GLV& 3177 & 1865\\\hline\hline

    Bandersnatch with GLV&  2621&1669\\\hline\hline
    \quad Endomorphism & 6 & 7 \\\hline
    \quad Scalar decomposition &  405 & 375\\\hline
    \quad multi scalar multiplication & 2176& 1285 \\\hline
    Overall Improvement & 21\%  & 10\% \\\hline
  \end{tabular}
  \caption{Bandersnatch vs Jubjub: Constraints count}
  \label{tab:r1cs_full}
\end{table*}


The \href{https://github.com/zhenfeizhang/bandersnatch}{Rust implementation} of the Bandersnatch scalar multiplication
algorithm confirms our estimations: the circuit for GLV method is $21\%$ smaller
than the Jubjub implementation with R1CS, and $10\%$ smaller with Plonk.

\subsubsection{R1CS}
We evaluate the number of constraints for a 
variable base group multiplication. For a double-and-add
algorithm, 
our code reports 3177 constraints in 
total.
As a sanity check, within the core logic,
it takes 6 constraints per addition, 5 constraints
per doubling and 2 constraints per bit selection. This adds
up to 13 constraints per bit, or 3177 constraints per
group multiplication (and we reasonably assume some system overhead
consumes another 10 constraints). 

Now, in the case of GLV,
% Let us first explain the obstacle of implementing the GLV
%with R1CS. T
the endomorphism is almost free: it requires 
6 constraints. The MSM inside the GLV can also be done 
with $2176$ constraints using the above double-then-add
techniques.
The difficult part is the circuit for scalar decomposition,
which is to prove $n = n_1 +\lambda n_2 \bmod r$ where
$n$ and $\lambda$ are around 256 bits,
$n_1$ and $n_2$ are around 128 bits, and
$r$
is the order of the scalar field.
We implemented this part of the code with $405$ constraints
via hand optimized circuits.

\subsubsection{Plonk}
We apply a similar analysis for our customized, ECC-friendly 
plonk system. In a double-and-add circuit, each bit in the scalar
takes 7 constraints in total, i.e., 2 for point selection, 2 
for addition, 2 for doubling, and 1 for binary decomposition.
For a scalar length of 256 bits, this is roughly $1800$ constraints.

In comparison, with GLV, per bit we account for 4 constraints for 2x point selections,
4 constraints for 2x point additions, and 2 constraints for doubling.
There is also a bit decomposition cost which was done during scalar
decomposition. The gain comes from the fact that our scalars are only
128 bits each.



\section{Conclusion}
Tne last decade has seen great improvements on practical zk-SNARK systems.
An essential stepping stone of these schemes is an efficient elliptic
curve whose base field matches the scalar field for some pairing-friendly curve.
On this note, we present Bandersnatch as an alternative to the commonly used
base curve Jubjub. Due to the existence of an efficiently computable
endomorphism, the scalar multiplication over this curve is 42\% times
faster than the Jubjub curve.
For multi scalar multiplications, we report a narrowed advantage over Jubjub 
curve when the dimension is small, but it vanishes for larger
dimensions.
We also do not observe any improvement in terms of number of constraints in 
the corresponding R1CS circuit.


% \bigskip
% \paragraph*{\textbf{Acknowledgments.}} We would like to thank Weikeng
% Chen, Luca De Feo, Justin Drake, Youssef El Housni, Dankrad Feist,
% Gottfried Herold and Daira Hopwood for fruitful discussions. 

\bibliography{bandersnatch,cryptobib/abbrev3,cryptobib/crypto}
\bibliographystyle{unsrt}
\end{document}
