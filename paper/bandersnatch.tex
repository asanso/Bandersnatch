\documentclass{amsart}

\usepackage{hyperref}
\usepackage[capitalize]{cleveref}
\usepackage{tikz}
\usepackage[all]{xy}
\usepackage[foot]{amsaddr}

%\newcommand{\AS}[1]{\color{red}#1\color{black}}

%%%%%%%%%%%%%%%%%%%%%%%%%%%%%%%%%%%%%%%%%%%%%%%%%%
%% Article 
%%%%%%%%%%%%%%%%%%%%%%%%%%%%%%%%%%%%%%%%%%%%%%%%%%

\begin{document}
\title[Bandersnatch]{Bandersnatch: a fast elliptic curve built over the BLS12-381 scalar field.}
\author{Simon Masson}
\address{Heliax}
\author{Antonio Sanso}
\address{Ethereum Foundation and Ruhr Universit{\"a}t Bochum}


\maketitle
\medskip
\begin{abstract}
 In this short note we  introduce Bandersnatch a new elliptic curve built over the BLS12-381 \cite{bls12381} scalar field. The curve is similar to JubJub \cite{jubjub} but is equipped with the GLV endomorphism \cite{10.1007/3-540-44647-8_11} hence it has faster scalar multiplication.
 \end{abstract}

\section{Introduction} 
BLS12-381 is a pairing friendly curve created by Sean Bowe in 2017 \cite{bls12381}. Currently BLS12-381 is universally used for digital signatures and zero-knowledge proofs by many project orbiting in the 
blockchin universe: Zcash, Ethereum 2.0, Skale, Algorand, Dfinity, Chia, and more. The ZCash team also introduced a new curve built over the BLS12-381 scalar field JubJub  \cite{jubjub}: a twisted Edwards curve that can be made efficient inside of the zk-SNARK circuit.
In order for some cryptographic application to scale it is needed to have a curve like JubJub but with faster scalar multiplication. One efficient way to speed scalar multiplication up is to employ the celebrated GLV endomorphism \cite{10.1007/3-540-44647-8_11} (also used by the “Bitcoin curve” - secp256k1). This technique was until few months ago protected by a US Patent that is now expired and freely usable.
We performed an exhaustive search of curves where the GLV endomorphism could be used over the BLS12-381 scalar field using the Complex Multiplication (CM) method of generating an elliptic curve. 

\section{Bandersnatch} 

We performed an exhaustive search of curves where the GLV endomorphism could be used over the BLS12-381 scalar field using the Complex Multiplication (CM) method of generating an elliptic curve. To be more specific we computed the order of such curves for the discriminants from $-1$ to $-388$.
We found one suitable curve for discriminant $-8$ with order 
$$2^2\cdot 13108968793781547619861935127046491459309155893440570251786403306729687672801$$
Bandersnatch is also twisted secure: the order of the twist is $$2^7 \cdot 3^3 \cdot 15172417585395309745210573063711216967055694857434315578142854216712503379$$
The curve has j-invariant equal $8000$ and exhibits $125.75$ bit security  . Given the shape of the order it can be expressed also in Montgomery and Edward form. 
Bandersnatch in twisted Edwards form looks like

$$-5x²+y² = 1+dx²y²$$ with $d=\frac{138827208126141220649022263972958607803}{171449701953573178309673572579671231137}$.

%Bandersnatch's endomorphism
\medskip
The endomorphism of degree 2 is defined by
$$\psi(x,y,z) = (xa_1(y+a_2z)(y+a_3z), b_1(y+b_2z)(y+b_3z)yz^2, (y+c_1z)(y+c_2z)yz^2)$$
and can be computed in 17 multiplications and 6 additions modulo $p$ ($a_i, b_i, c_i$ are integers modulo $p$).

%### Scalar multiplication improvement
From the efficient endomorphism $\psi$, it is easy to apply the GLV method and improve the scalar multiplication cost:
Roughly, a scalar multiplication $[n]P$ cost $(\log n) \text{Dbl} + (\log n/2) \text{Add}$.
Using the GLV endomorphism, we can compute $[n]P$ using $(\log n/2 )\text{Dbl} + (3\log n/8) \text{Add}$, plus few precomputations.

We obtained `python` benchmarks between the double-and-add algorithm and the GLV method applied in the case of our curve, and the GLV version is about 30\% faster 
\footnote{Source code \url{https://github.com/asanso/Bandersnatch/}.}

\bigskip
\paragraph*{\textbf{Acknowledgments.}} we would like to thank Luca De Feo, Justin Drake, Dankrad Feist, Daira Hopwood and Zhenfei Zhang for fruitful discussions.

\bibliography{bandersnatch}{}
\bibliographystyle{unsrt}
\end{document}
